\documentclass[12pt,letterpaper]{hmcpset}
\usepackage[margin=1in]{geometry}
\usepackage{graphicx}
\usepackage{amssymb}
% info for header block in upper right hand corner
\name{Name: \underline{\hspace{3cm}}}
\class{Math 40, Section \underline{\hspace{1cm}}}
\assignment{HW04 - Matrix Operations and Matrix Inverses }
\duedate{February 9, 2017 }

\begin{document}
\section*{}
\problemlist{Section 3.2 Numbers; 4, 22 \\ Section 3.3 Numbers; 13a, 13b, 22, 47}


\problemlist{}

\begin{problem}[3.2.4]
\textit{Given $A$ and $B$ solve the equation, $2(A-B+X)=3(X-A)$ ,  for$X$}
$$ A= \begin{bmatrix}
1&2\\3&4
\end{bmatrix} \text{and } B=\begin{bmatrix}
-1&0\\1&1
\end{bmatrix}
$$
\end{problem}

\begin{solution}

\end{solution}
\newpage
\begin{problem}[3.2.22]
\textit{Prove that, for square matrices $A$ and $B$ , $AB=BA$ if and only if $(A-B)(A+B) = A^2-B^2$.}
\end{problem}

\begin{solution}

\end{solution}
\newpage
\begin{problem}[3.3.13]
$$\textit{Let} A=\begin{bmatrix}
1&2\\2&6
\end{bmatrix}, \textbf{b}_1=\begin{bmatrix}
3\\5
\end{bmatrix}
,\textbf{b}_2=\begin{bmatrix}
-1\\2
\end{bmatrix}
\textbf{b}_3=\begin{bmatrix}
2\\0
\end{bmatrix}
$$
\begin{enumerate}
\item[a.]\textit{Find $A^{-1}$ and use it to solve the three systems $Ax=\textbf{b}_1$ , $Ax=\textbf{b}_2 ,$ and $Ax=\textbf{b}_3$.}
\item[b]\textit{Solve all three systems at the same time by row reducing the augmented\\ matrix $\left[ A | \textbf{b}_1 \textbf{b}_2 \textbf{b}_3\right]$ using Gauss-Jordan elminiation.}
\end{enumerate}
\end{problem}

\begin{solution}

\end{solution}
\newpage
\begin{problem}[3.3.22]
\textit{Solve the given matrix equation for X. Simplify your answers as much possible. (In the words of Albert Einstein, "Everything should be made as simple as possible, but not simpler") Assume that all matricies are invertible}
$$ (A^{-1}X)^{-1}= A(B^{-2}A)^{-1}$$
\end{problem}
\newpage
% Add pairs of problems and solutions as needed
\begin{problem}[3.3.47]
\textit{Prove that if A and B are square matricies and AB is invertible, then both A and B are invertible.}
\end{problem}

\end{document}
